
\newpage
\begin{appendices}
%-----

\renewcommand\thechapter{}
\chapter[Ejercicios del Cap\'{\i}tulo~\ref{CHAP07}]{Ejercicios del 
                        Cap\'{\i}tulo~\ref{CHAP07}}
\label{ejercicios:chap07}
\stepcounter{mychapter} % INCREASE COUNTER BY 1

\begin{prob}
\label{chap07:prob1}
Ejecutar los comandos de las celdas \mytext{In [6]:-In [10]:}
de la sesión {\ipython} en la 
página~\pageref{code:chap07:matplotlib:ex:01}  
en secuencias diferentes a la presentada en esa sesión {\ipython}.
\end{prob}
%
\begin{prob}
\label{chap07:prob2}
Realizar los ejercicios propuestos durante la discusión
de los ejemplos de las secciones~\ref{chap:07:matplotlib:ex1}
(en la página~\pageref{chap:07:matplotlib:ex1}), \ref{chap:07:matplotlib:ex2} 
(en la página~\pageref{chap:07:matplotlib:ex2}) y \ref{chap:07:matplotlib:ex3}
(en la página~\pageref{chap:07:matplotlib:ex3}).
\end{prob}
%
\begin{prob}
\label{chap07:prob3}
Ejecutar los comandos de las celdas \mytext{In [6]:-In [11]:}
de la sesión {\ipython} en la 
página~\pageref{code:chap07:matplotlib:ex:02}  
en secuencias diferentes a la presentada en esa sesión {\ipython}.
\end{prob}
%
\begin{prob}
\label{chap07:prob5}
Modificando el programa de la 
página~\pageref{code:chap06:archivooutput:ex:01}
e incluyendo instrucciones
para graficar, 
reproducir una de las gráficas que se presentan en la 
figura~\ref{fig:chap06:flores:01}, en la
página~\pageref{fig:chap06:flores:01}.
\end{prob}
%
\begin{prob}
\label{chap07:prob6}
Con apoyo en la figura figura~\ref{fig:chap07:ArregloSubCuadros},
en la página~\pageref{fig:chap07:ArregloSubCuadros}, hacer modificaciones
al programa de la página~\pageref{code:chap07:matplotlib:ex:03} 
para obtener cuatro subcuadros.
\end{prob}
%
\begin{prob}
\label{chap07:prob7}
En el programa que se muestra en la 
página~\pageref{code:chap07:matplotlib:ex:03},
desactivar (añadiendo el caracter \url{#} 
al inicio
de la línea) o eliminar la instrucción \mytext{\url{fig.tight_layout()}} 
(en la línea de código \mytextb{22}) y ejecutar el programa para 
observar su efecto. Buscando recuperar parcialmente uno de los ajustes
que esa opción produce, ¿qué valor se le asignaría a la variable
\mytext{loc} (ver opciones en la tabla~\ref{chap:07:tabla:formatoleyenda},
en la página~\pageref{chap:07:tabla:formatoleyenda})
 en la línea de código \mytextb{13}?.
\end{prob}
%
\begin{prob}
\label{chap07:prob8}
Con apoyo en los ejercicios~\ref{chap07:prob5}~y~\ref{chap07:prob6},
reproducir la figura
\ref{fig:chap06:flores:01}, en la
página~\pageref{fig:chap06:flores:01}.
\end{prob}
%
\begin{prob}
\label{chap07:prob9}
Para corroborar que en las líneas de código \mytextb{26-28} del
programa de la página~\pageref{code:chap07:matplotlib3D:prog:01}
se asignan objetos de tipo \mytext{numpy.ndarray} y dimensiones
\mytext{16} filas por \mytext{16} columnas, después
de la línea de código \mytextb{24} inserte las instrucciones:
\begin{mymdframed}
\lstset{numberfirstline=true, firstnumber=26, numbers=left, 
        numberstyle=\scriptsize, stepnumber=1, numbersep=3pt}
\begin{lstlisting}[basicstyle=\ttfamily\scriptsize,frame=single] 

print('El objeto asignado a la variable X es de tipo: {0}'.format(type(X)))
print('El objeto asignado a la variable X tiene dimensiones: {0}'.
                                                    format(np.shape(X)))
\end{lstlisting}
\end{mymdframed}

\begin{mymdframed}
\lstset{numberfirstline=true, firstnumber=32, numbers=left, 
        numberstyle=\scriptsize, stepnumber=1, numbersep=3pt}
\begin{lstlisting}[basicstyle=\ttfamily\scriptsize,frame=single] 
print('El objeto asignado a la variable x es de tipo: {0}'.
                                                    format(type(x)))
print('El objeto asignado a la variable x tiene dimensiones: {0}'.
                                                    format(np.shape(x)))

\end{lstlisting}
La función {\numpy} \mytext{shape} se describe en
(\url{http://docs.scipy.org/doc/numpy-1.10.1/reference/generated/numpy.ndarray.shape.html}).
\end{mymdframed}
%
\end{prob}
%
\begin{prob}
\label{chap07:prob10}
En la sesión {\ipython} de la página~\pageref{code:chap07:matplotlib3D:02:in},
ejecutar las instrucciones \mytext{X.flatten()}, 
\mytext{Y.flatten()} y \mytext{Z.flatten()} para precisar lo que generan
(aunque el resultado debe ser evidente de la línea de código
\mytextb{14} del programa en 
la página~\pageref{code:chap07:matplotlib3D:prog:02}). 
\end{prob}
%
\begin{prob}
\label{chap07:prob11}
Después de
agregar el símbolo numeral (\mytext{\#}) al inicio
de la línea de código \mytextb{15} del programa en la
página~\pageref{code:chap07:matplotlib3D:prog:02} para convertirla
en un comentario, ejecutar 
el programa y observar el efecto que ello causa en la gráfica 
que se genera comparándola con la figura de la 
página~\ref{fig:chap07:matplotlib:3D:ex:02}.
\end{prob}
%
%\begin{prob}
%\label{chap07:prob7}
%\end{prob}

\end{appendices}

